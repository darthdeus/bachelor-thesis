\chapter{Experiments}

To verify our implementation we conducted an experiment with N \todo{cislo}
computer science students which each played 20 games and answered questions
to measure the strength of our MCTS AI, the balance of the generated encounters,
and the overall game difficulty.

\missingfigure{grafy}

The results show that most participants consider our MCTS AI to be strong
enough to provide a challengem which is also proven by the 40\% winrate of
the participants. \todo{aktualizovat cisla}

Given the number of played games \todo{overit, ze jich mame dost} we can
rule out the AI having a better setup in most of the scenarios. Based on the
results, around 35\% \todo{aktualizovat cisla} of the games were balanced,
which means in the resulting 65\% one of the players had an advantage.

We consider this result to be positive as the games were generated without
any human intervention and weren't altered before running the experiment.
\todo{link na appendinx s instrukcema na pregenerovani experimentu}

Considering that 35\% of the generated games are balanced, this would allow
the algorithm to be used offline as is to aid design of game levels with
some manual checking of the resulting games.

If one were to design the encounter completely from scratch to be balanced,
it would be very difficult given the number of variables that need to be
optimized.

\subsection{TODO}

\begin{description}[align=right,labelwidth=3cm]
\item bud zminit ze by to slo delat online s nejaky additional checkem
\item nebo zlepsit fitness aby vyresila patologicke pripady z experimentu?
\end{description}

