\chapter{Problem Definition}

Tactical turn-based games like \citet{duelyst} and \citet{faeria} have become
very popular in the recent years.

More content can easily be added to the game (new spells, monsters, etc.),
but it is difficult to make sure the game stays balanced as the new content
is added over time. This is generally done by human play-testing \todo{reference}
in-house before the game updates are released, and takes large amounts of time.

This thesis aims to solve the problem with \emph{procedural content generation}
(PCG).  We define PCG as \emph{the algorithmic creation of game content with
little or no user input} \citep{pcgbook} We introduce a custom game similar to
\citet{duelyst}, and show that search based PCG can create new content that is
balanced.

\section{Game types}

We narrow our scope to turn-based games, since simultaneous move games provide
an additional challenge when building an AI, which for us is just a means to help
with PCG.

\section{Types of PCG currently being used}

Many games utilize different kinds of PCG these days. Large comercial games
such as \citet{diablo} use PCG to generate unique map layout of certain areas.
Some games like the popular \citet{minecraft} go to an extreme and procedurally
generate the whole world. But PCG is not only limited to large games, even
small indie games like \citet{spelunky} or \citet{terrarira} utilize
PCG for generating playable levels.

Games also use PCG for item creation, where the specific attributes of an item are generated
online (e.g. Borderlands \citet{borderlands}).

\section{Thesis Goals}

Our main goal for this thesis is to generate balanced encounters. For this, we decided
to implement a custom game with flexible game mechanics so that there are many different
ways to create a balanced encounter. The game is turn-based, zero-sum, with perfect information.
\todo{see chap2 for mechanics}

We implement the game both in the form of a simulator that can be used as a library,
and a GUI that a human player can use to play the game and test it. We also implement
an AI for the game so that we can automatically evaluate and test games in our PCG algorithm.


cite \cite{Genberget08}

citet \citet{Genberget08}

citet-star \citet*{Genberget08}

citep \citep{Genberget08}

citep-star \citep*{Genberget08}

