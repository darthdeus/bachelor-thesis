\chapter{Game Mechanics}

\section{Motivation for a Custom Game}

- game that can generalize to real life scenarios
- simple mechanics that are representative
- fast to simulate

\section{Game Mechanics in Detail}

Two teams fight on a hexagonal map (\emph{arena}) of small size (radius of
5-10 hexes). Each team consists of a small number of player controller
characters (\emph{mages} for short), and each mage has a small number of
\emph{abilities}, \emph{health}, and \emph{action points}. Players take turns, during
which each player has control over one of his mages. The player can issue
commands to move around the arena and use abilities. Both movement and ability
usage costs action points, which are restored at the end of the turn.  Moving
one hex costs one action point. The cost of using an ability varies, and is one
of the parameters that we optimize for when looking for a balanaced game. An
ability can also have a \emph{cooldown}, which prevents repeated usage for a
given number of turns \todo{turn}. Note that the cooldown can
be zero, which means the ability can be used multiple times per turn
\todo{turn}.

\missingfigure{obrazek areny?}

Abilities can do direct damage to an enemy mage, apply a \emph{debuff} (causing
damage and decreasing action points over time), and create an area of effect
(\emph{AOE}) debuff that spans multiple hexes in the arena. Both the debuffs and
AOEs are applied at the end of each turn \todo{rozlisovat turn maga a turn hry
(round?)}, which happens after both players finish playing all their mages.
Both debuffs and AOEs also have a lifetime, which specifies how many turns
\todo{znova turn} the effect lasts.

\section{Simulator}
