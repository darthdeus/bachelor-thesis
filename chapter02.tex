\chapter{Game Rules and Mechanics}

For the purposes of this thesis we'll only stick to turn-based gameplay. We
still wanted our game to be general enough so that a wide variety of
playstyles can be represented within the game rules.

Two teams fight on a hexagonal map (\emph{arena}) of small size (radius of
5--10 hexes). Each team consists of a small number of player controller
characters (\emph{mages} for short), and each mage has a small number of
\emph{abilities}, \emph{health}, and \emph{action points}. Players take turns, during
which each player has control over one of his mages. The player can issue
commands to move around the arena and use abilities. Both movement and ability
usage costs action points, which are restored at the end of the turn.  Moving
one hex costs one action point. The cost of using an ability varies, and is one
of the parameters that we optimize for when looking for a balanaced game. An
ability can also have a \emph{cooldown}, which prevents repeated usage for a
given number of turns \todo{turn}. Note that the cooldown can
be zero, which means the ability can be used multiple times per turn

Abilities can do direct damage to an enemy mage, apply a \emph{debuff} (causing
damage and decreasing action points over time), and create an area of effect
(\emph{AOE}) debuff that spans multiple hexes in the arena. Both the debuffs and
AOEs are applied at the end of each turn \todo{rozlisovat turn maga a turn hry
(round?)}, which happens after both players finish playing all their mages.
Both debuffs and AOEs also have a lifetime, which specifies how many turns
\todo{znova turn} the effect lasts.

The motivation for having cooldowns is that it allows us to generate powerful
abilities that don't necessarily take up the whole turn of the player by costing
a lot of action points. If the ability is cheap, but has a large cooldown, it can
serve a strategic purpose, as the player might want to prepare his position in order
to use the ability when the cooldown wears off.

AOE abilities also improve positional gameplay, as they can be used to force the enemy
player to move out of position.

\missingfigure{obrazek areny?}

\section{Simulator}

Our game is represented as a single \emph{state} object onto which \emph{actions} can be applied
to transition to the next state. The possible actions are:

\begin{description}
\item [AbilityUse] Use an ability targeting an enemy that is already in range.
\item [Move] Move the current mage to a different hex on the map.
\item [EndTurn] Finish the current turn.
\end{description}

There are also a few special action types that:

\begin{description}
\item [DefensiveMove] Serves the same purpose as \emph{Move}, but carries an
additional information in the sense that \emph{DefensiveMove} can only be
the last action of the turn.
\item [AttackMove] Combines the \emph{Move} and \emph{AbilityUse} actions into one.
\item [NullAction] Doesn't do anything and is mostly used as a placeholder in cases no action is possible.
\end{description}
