\chapter{Analysis of Game Mechanics}

Two teams fight on a hexagonal map (\emph{arena}) of small size (radius of 5\-10
hexes).  \unsure{screenshot mapy?} Each team consists of a small number of
player controller characters (\emph{mages} for short), and each mage has a small
number of \emph{abilities}, health, and \emph{action points}. Players take turns,
during which each player has control over one of his mages. The player can
issue commands to move around the arena and use abilities. Both movement and
ability usage costs action points, which are restored at the end of the turn.
Moving one hex costs one action point. The cost of using an ability varies, and
is one of the parameters that we optimize for when looking for a balanaced
game. An ability can also have a \emph{cooldown}, which prevents repeated usage
for a given number of turns \todo{znovu rozlisovat turn}. Note that the
cooldown can be zero, which means the ability can be used multiple times per
turn \todo{turn}.

Abilities can do direct damage to an enemy mage, apply a \emph{debuff} (causing
damage and decreasing action points over time), and create an area of effect
(\emph{AOE}) debuff that spans multiple hexes in the arena. Both the debuffs and
AOEs are applied at the end of each turn \todo{rozlisovat turn maga a turn hry
(round?)}, which happens after both players finish playing all their mages.
Both debuffs and AOEs also have a lifetime, which specifies how many turns
\todo{znova turn} the effect lasts.


\todo{popsat druhy akci}
\todo{popsat jak funguje simulator}
\todo{popsat strukturu}


\section{High level actions of Monte-Carlo Tree Search}

After some experimentation, we've settled down for three high level actions
that represent most of what a player might want to do.

\begin{description}[align=right,labelwidth=3cm]
\item [AbilityUse] Use an ability targeting an enemy that is already in range.
\item [AttackMove] Move into the range of an enemy and use an ability.
\item [DefensiveMove] Look for a place on the map that is not visible to the enemy and move there (to avoid damage).
\end{description}

Combined actions help significantly reduce the depth of the game tree. Most
prominent is the fact that we don't allow arbitrary \emph{Move} actions, but
only \emph{DefensiveMove}.

\section{Title of the second subchapter of the second chapter}

\chapter{Encounter balancing}

In order to evaluate the balance of a matchup, we came up with the following fitness function.


During the experiment we encountered multiple ways ES tried to exploit the game
mechanics to maximize the balance fitness function in ways that were
undesirable. One example being when the resulting games end up being short as
the algorithm generates mages with lots of AOE abilities that cover the whole
map in the first turn, resulting in immediate death of all characters.

We balance this by introducing an additional fitness function for game length
in the form of a cumulative normal distribution with $\sigma = 10$ and $\mu =
3$.\unsure{overit jestli to tak fakt je}.  This prevents ES from moving towards
short games.

\section{Evolution Strategies}



\section{Experiment}
