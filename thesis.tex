%%% The main file. It contains definitions of basic parameters and includes all other parts.

%% Settings for single-side (simplex) printing
% Margins: left 40mm, right 25mm, top and bottom 25mm
% (but beware, LaTeX adds 1in implicitly)
\documentclass[12pt,a4paper]{report}
\setlength\textwidth{145mm}
\setlength\textheight{247mm}
\setlength\oddsidemargin{15mm}
\setlength\evensidemargin{15mm}
\setlength\topmargin{0mm}
\setlength\headsep{0mm}
\setlength\headheight{0mm}
% \openright makes the following text appear on a right-hand page
\let\openright=\clearpage

%% Settings for two-sided (duplex) printing
% \documentclass[12pt,a4paper,twoside,openright]{report}
% \setlength\textwidth{145mm}
% \setlength\textheight{247mm}
% \setlength\oddsidemargin{14.2mm}
% \setlength\evensidemargin{0mm}
% \setlength\topmargin{0mm}
% \setlength\headsep{0mm}
% \setlength\headheight{0mm}
% \let\openright=\cleardoublepage

%% Prefer Latin Modern fonts
\usepackage{lmodern}

%% Further useful packages (included in most LaTeX distributions)
\usepackage{amsmath}        % extensions for typesetting of math
\usepackage{amsfonts}       % math fonts
\usepackage{amsthm}         % theorems, definitions, etc.
\usepackage{bbding}         % various symbols (squares, asterisks, scissors, ...)
\usepackage{bm}             % boldface symbols (\bm)
\usepackage{graphicx}       % embedding of pictures
\usepackage{fancyvrb}       % improved verbatim environment
\usepackage{natbib}         % citation style AUTHOR (YEAR), or AUTHOR [NUMBER]
\usepackage[nottoc]{tocbibind} % makes sure that bibliography and the lists
			    % of figures/tables are included in the table
			    % of contents
\usepackage{dcolumn}        % improved alignment of table columns
\usepackage{booktabs}       % improved horizontal lines in tables
\usepackage{paralist}       % improved enumerate and itemize
\usepackage[usenames]{xcolor}  % typesetting in color

\usepackage[colorinlistoftodos,prependcaption,textsize=tiny]{todonotes}
\usepackage{xargs}
\newcommandx{\unsure}[2][1=]{\todo[linecolor=red,backgroundcolor=red!25,bordercolor=red,#1]{#2}}
\newcommandx{\change}[2][1=]{\todo[linecolor=blue,backgroundcolor=blue!25,bordercolor=blue,#1]{#2}}
\newcommandx{\info}[2][1=]{\todo[linecolor=OliveGreen,backgroundcolor=OliveGreen!25,bordercolor=OliveGreen,#1]{#2}}
\newcommandx{\improvement}[2][1=]{\todo[linecolor=Plum,backgroundcolor=Plum!25,bordercolor=Plum,#1]{#2}}
\newcommandx{\thiswillnotshow}[2][1=]{\todo[disable,#1]{#2}}

\usepackage{enumitem}

%% Generate PDF/A-2u
\usepackage[a-2u]{pdfx}

%% Character encoding: usually latin2, cp1250 or utf8:
\usepackage[utf8]{inputenc}

\usepackage{hyperref}


%%% Basic information on the thesis

%%% XXX TODO
\title{HexMage}

% Thesis title in English (exactly as in the formal assignment)
\def\ThesisTitle{Using procedural content generation to balance encounters in RPG-like games}

% Author of the thesis
\def\ThesisAuthor{Jakub Arnold}

% Year when the thesis is submitted
\def\YearSubmitted{2017}

% Name of the department or institute, where the work was officially assigned
% (according to the Organizational Structure of MFF UK in English,
% or a full name of a department outside MFF)
\def\Department{Name of the department}

% Is it a department (katedra), or an institute (ústav)?
\def\DeptType{Department}

% Thesis supervisor: name, surname and titles
\def\Supervisor{Mgr. Jakub Gemrot}

% Supervisor's department (again according to Organizational structure of MFF)
\def\SupervisorsDepartment{department}

% Study programme and specialization
\def\StudyProgramme{study programme}
\def\StudyBranch{study branch}

% An optional dedication: you can thank whomever you wish (your supervisor,
% consultant, a person who lent the software, etc.)
\def\Dedication{%
Dedication.
}

% Abstract (recommended length around 80-200 words; this is not a copy of your thesis assignment!)
\def\Abstract{%
Procedural content generation (PCG) is mostly examined in the context of
map/environment creation, rather than generating the actual game characters.
The topic of this thesis is to explore PCG options for balancing encounters in
turn-based RPG-like games. TODO {prepsat podle cyrilovych poznamek} Part of the
work is also designing a simple game with flexible enough mechanics so that
different approaches to PCG can be tested easily. The game consists of a
hex-based arena in which two teams fight, each consisting of a few charactes
with unique abilities. We test the generated team setup using simulated AI vs
AI combat, which shows that for most initial configurations there exists an
equally strong, but different opponent.}

% 3 to 5 keywords (recommended), each enclosed in curly braces
\def\Keywords{%
{video games} {encounter balancing} {hex arena} {rpg elements}
}

%% The hyperref package for clickable links in PDF and also for storing
%% metadata to PDF (including the table of contents).
%% Most settings are pre-set by the pdfx package.
\hypersetup{unicode}
\hypersetup{breaklinks=true}

% Definitions of macros (see description inside)
\include{macros}

% Title page and various mandatory informational pages
\begin{document}
\include{title}

%%% A page with automatically generated table of contents of the bachelor thesis

\tableofcontents

%%% Each chapter is kept in a separate file
\chapter*{Introduction}
\addcontentsline{toc}{chapter}{Introduction}

An increasing number of computer games is using procedural content generation (PCG) as one of their core mechanics. This can be in different contexts, most commonly for generating new levels (Diablo [Xcitace]). Occasionally games even generate player collectible items (Borderlands [Xcitace]), but there has not been much research on the use of procedural generation for balancing encounters in RPG games. By this we mean procedurally generating enemies that can be defeated by the player, but pose a challenge. A crucial criteria here is that the balance is not simply achieved by creating the enemy as an exact clone of the player, but rather explore the search-space to find an enemy that is not only balanced, but also different from the player.

One particular application for this kind of PCG is automatic difficulty adjustments based on the player's skill. Another possible use could be automatic generation of new and unique enemies based on given constraints, which is the approach we chose in this thesis.

We have created a custom game with mechanics that are simple enough to simulate quickly, yet flexible enough to represent a large search space. There are two teams that fight in a hexagonal arena, each consists of a small number of player controlled characters (mages), and each mage has a small number of abilities. In each turn the player has control over one of his characters, and both move around the map and cast spells, in any order he wishes. The only limit is the number of action points the character has available, which are consumed both by movement and ability usage. The side that first eliminates the opposing team wins.

All information is visible to all players, and all actions are completely deterministic. There is no time limit for the player action, which means the player could theoretically calculate a perfect move given enough time.

The goal of this thesis is to explore PCG options for balancing encounters in turn-based RPG-like games. We design a simple game with flexible mechanics, and build an AI that can be used to approximate the player. We then use genetic algorithms to evolve balanced opponents for the given player team, using AI vs AI combat as a fitness function.

\subsection*{Organization}

In Chapter 1 we begin by defining the scope of the work, the used methods, and list related work. Chapter 2 follows by exploring our game mechanics in detail and explaining the choices behind them. Next in Chapter 3 we will go over implementation details of the methodology, such as how the game simulator is implemented, the choice of technology, and algorithm choices in the experiments.

Chapter 4 describes what experiments have been done with results discussed in Chapter 5. Lastly, Chapter 6 serves as the user and developer documentation, describing implementation detail not relevant to the methodology mentioned in Chapter 3. Chapter 7 closes with a conclusion and possibilities for future work.
% \part{Analysis}
\chapter{Problem Definition}

Tactical turn-based games like Duelyst and Faeria \todo{reference link}
have become very popular in the recent years.

More content can easily be added to the game (new spells, monsters, etc.),
but it is difficult to make sure the game stays balanced as the new content
is added over time. This is generally done by human play-testing \todo{reference}
in-house before the game updates are released, and takes large amounts of time.

This thesis aims to solve the problem with PCG. We introduce a custom
game similar to Duelyst, etc., and show that search based PCG can create
new content that is balanced.

\section{Game types}

\section{Types of PCG currently being used}

\section{Thesis Goals}

Our main goal for this thesis is to generate balanced encounters. For this, we decided
to implement a custom game with flexible game mechanics so that there are many different
ways to create a balanced encounter. The game is turn-based, zero-sum, with perfect information.
\todo{see chap2 for mechanics}

We implement the game both in the form of a simulator that can be used as a library,
and a GUI that a human player can use to play the game and test it. We also implement
an AI for the game so that we can automatically evaluate and test games in our PCG algorithm.


cite \cite{Genberget08}

citet \citet{Genberget08}

citet-star \citet*{Genberget08}

citep \citep{Genberget08}

citep-star \citep*{Genberget08}


% \part{Implementation}
\chapter{Game Rules and Mechanics}
\label{chapter02}|

For the purposes of this thesis we'll only stick to turn-based gameplay. We
still wanted our game to be general enough so that a wide variety of
playstyles can be represented within the game rules.

Two teams fight on a hexagonal map (\emph{arena}) of small size (radius of
5--10 hexes). Each team consists of a small number of player controller
characters (\emph{mages} for short), and each mage has a small number of
\emph{abilities}, \emph{health}, and \emph{action points}. Players take turns, during
which each player has control over one of his mages. The player can issue
commands to move around the arena and use abilities. Both movement and ability
usage costs action points, which are restored at the end of the turn.  Moving
one hex costs one action point. The cost of using an ability varies, and is one
of the parameters that we optimize for when looking for a balanaced game. An
ability can also have a \emph{cooldown}, which prevents repeated usage for a
given number of rounds. Note that the cooldown can be zero, which means the
ability can be used multiple times per turn

Abilities can do direct damage to an enemy mage, apply a \emph{debuff} (causing
damage and decreasing action points over time), and create an area of effect
(\emph{AOE}) debuff that spans multiple hexes in the arena. Both the debuffs and
AOEs are applied at the end of each round. Both debuffs and AOEs also have a lifetime,
which specifies how many rounds the effect lasts.

There is also an important distinction between a \emph{turn} and a \emph{round}.
A turn means playing all the actions a single mage can do with his action points,
and ends when the player decides he is finished playing with that one single character.
A round ends when all of the characters have played their turn, and it is at that point
when debuffs, AOEs, cooldowns and action points are re-calculated (see below).

The motivation for having cooldowns is that it allows us to generate powerful
abilities that don't necessarily take up the whole turn of the player by costing
a lot of action points. If the ability is cheap, but has a large cooldown, it can
serve a strategic purpose, as the player might want to prepare his position in order
to use the ability when the cooldown wears off. AOE abilities also improve positional
gameplay, as they can be used to force the enemy player to move out of position.

\missingfigure{obrazek areny?}

\section{Simulator}

Part of our game is a simulator that can be used as a library and encapsulates
all of the game rules and mechanics. This is then used by both the AI and the
PCG algorithm and can thus be run separately from the main game. The game is
internally represented by a \emph{state} object, and all of the possible player
actions are encapsulated by an \emph{action} object.  Playing the game through
the library then simply becomes a matter of applying a state transition
function $f: (\text{state}, \text{action}) \rightarrow \text{state}$. The
simulator also verifies that no invalid actions are applied through a thorough
list of invariant checks. These are automatically turned off in a release build
to make the simulator run as fast as possible.

The simulator is also built to be high performance and can easily run hundreds
of thousands to millions of actions per second on a consumer-grade PC. 
The state object is split into two parts, one that handles the
general immutable information that doesn't change as the game progresses (i.e.
max hitpoints, ability definitions, etc.), and one that handles all of the
mutable data, such as current hitpoints, current positions on the map, etc.
This allowed us to make state copies very fast as well, running only at a few
microseconds per copy.

\missingfigure{graf}

\subsection{Actions}
The possible actions are:

\begin{description}
\item [AbilityUse] Use an ability targeting an enemy that is already in range.
\item [Move] Move the current mage to a different hex on the map.
\item [EndTurn] Finish the current turn.
\item [DefensiveMove] Serves the same purpose as \emph{Move}, but carries an
additional information in the sense that \emph{DefensiveMove} can only be
the last action of the turn.
\item [AttackMove] Combines the \emph{Move} and \emph{AbilityUse} actions into one.
\item [NullAction] Doesn't do anything and is mostly used as a placeholder in cases no action is possible.
\end{description}

\chapter{Generating Encounters}

\section{Evolution Strategies}

\section{Choice of the Fitness Function}

In order to evaluate the balance of a matchup, we came up with the following fitness function.

During the experiment we encountered multiple ways ES tried to exploit the game
mechanics to maximize the balance fitness function in ways that were
undesirable. One example being when the resulting games end up being short as
the algorithm generates mages with lots of AOE abilities that cover the whole
map in the first turn, resulting in immediate death of all characters.

We balance this by introducing an additional fitness function for game length
in the form of a cumulative normal distribution with $\sigma = 10$ and $\mu =
3$.\unsure{overit jestli to tak fakt je}.  This prevents ES from moving towards
short games.

\chapter{Generating Encounters}
\label{chapter04}

We approach generating encounters as a search based problems with two different
approaches, Simulated Annealing \citep{ai-modern} and Evolution Strategies
\citep{evolution-strategies}.

To make the search algorithms as general as possible, we serialize our
internal game representation into a single vector of normalized floating
point values (DNA). The algorithm then doesn't need to understand our game
mechanics and restriction, and can be applied to different configurations of
the game independently.

To avoid balancing via specific attributes, one can simply remove them from
the serialization/deserialization routines that transform the game setup into
the DNA vector, without altering any of the logic for balancing encounters.

In the case of our experiments, we chose to stick with 2v2 games on a fixed map,
where each mage has only two abilities. This was chosen both with respect to our
questionnaire, and running time of the algorithms. Choosing a larger team setup or
a bigger map (or many different maps) would be difficult for the participants,
and would also take much longer to compute our experimental data.

\missingfigure{mapa}

\todo{popsat jak funguje reprezentace DNA?}

\section{Simulated Annealing and Evolution Strategies}

Our initial implementation of generating the encounters was using Simulated
Annealing. However, we had difficulty converging to good results \todo{fuj,
napsat jinak}.

As a result, we ran an experiment to sample the search space at roughly 20
million different points, and measured the change in fitness in the
neighbourhood of each point. We found that in each point's neighbourhood,
there are roughly 5 times more points that have worse fitness than the ones
that are an improvement over the current point. We also found that most of
these downward changes were much steeper between 4--7x than the improving
points. Our suspicion is that this is the main cause of failure of
Simmulated Annealing, which simply fails to find the upward slope.

For this reason we chose to try another approach, specificially Evolution
Strategies. In short, we take the initial DNA vector, generate around 50
neighbours in the search space, evaluate their fitness and take their weighted
sum as the new current DNA\@. This process is iterated until a DNA with
suitable fitness is found. We have found this approach to consistently converge
much faster than simulated annealing.

\section{Choice of the Fitness Function}

In order to evaluate the balance of a matchup, we evaluated the following three
criteria:

\begin{description}
\item [Balance] Unsurprisingly, part of the fitness function is comparing the strength of both teams against each other.
\item [Game length] We also put an interval restriction on game length.
\item [Team difference] Lastly, since we don't to create balance by making both teams identical, we added a third criteria
that measures the difference of both teams.
\end{description}

During the experiment we encountered multiple ways ES tried to exploit the
game mechanics to maximize the balance fitness function in ways that were
undesirable. One example being when the resulting games end up being short
as the algorithm generates mages with lots of AOE abilities that cover the
whole map in the first turn, resulting in immediate death of all characters.

We balance this by introducing an additional fitness function for game length
in the form of a cumulative normal distribution with mean around $10$ turns.

Lastly, the team difference is measured as an euclidean distance between the DNA
vectors of each team. Again, we use a cumulative normal distribution to put a lower
bound on the team difference.

\chapter{Experiments}
\label{chapter05}

\section{Comparing AIs against each other}

To compare the strength of our AIs against each other, we simulated 1000 randomly generated games
and had the AIs play against each other. Both teams were generated at random and there was no step taken
to make them balanced. Instead, each AI played both sides of the match. See \hyperref[tab:winrates]{Table 5.1} for results.

\begin{table}[h]
	\centering
	\begin{tabular}{|| c | c c c ||}
		\hline
		& MCTS & Rule & Random \\
		\hline\hline
		MCTS & N/A & 63\% & 88\% \\
		Rule & 37\%& N/A & 82\% \\
		Random & 12\% & 18\% & N/A \\
		\hline
	\end{tabular}
	\label{tab:winrates}
	\caption{Table showing win percentages over 1000 games between our different AI implementations.
	MCTS beats Random 88\% of the time, MCTS beats Rule 63\% of the time, and Rule beats Random 82\% of the time.}
\end{table}

As we can see, both MCTS and the Rule Based AI can beat our Random AI with a significant margin,
and MCTS can also consistently beat the Rule Based AI, despite playing for both generated teams.
From this we conclude that:

\begin{enumerate}
	\item A greedy aggressive strategy doesn't always win, which means the game mechanics are rich enough
	to reward strategic thinking.
	\item Thinking ahead (as MCTS does) provides enough value to be significant in terms of win rate.
\end{enumerate}

\missingfigure{vysledky jiny nez experiment}

\section{Participants}

The participants were all computer science students. All of them had at least
some experience with computer games, and were presented the mechanics of HexMage
before conducting the experiment.

\section{Experiment Design}

The goal of the experiment is to measure two things. One, if the generated
encounters are actually balanced. And two, if the AI is strong enough to pose a
challenge to the player. We will measure the overall winrate of the players,
and also compare how the AI did in games the player considered balanced.

The experiment consists of 20 different games of HexMage. All of the games are
played on the same map that was hand-designed beforehand. This was to allow the
participants to better get familiar with the game and think ahead. The games
are structured so that each player has 2 mages, each with 2 abilities. This was
to reduce the cognitive overhead for the participants and allow them to more easily
adjust to the 20 completely different scenarios.

The first 10 of the 20 games had the player team hand designed, and the
opponent (played by the AI) generated with our PCG algorithm. The remaining 10
games had both teams generated with no manual tweaks or changes. Having some of
the games hand designed allows us to show that the PCG algorithm can balance
against constraints that aren't completely random. A hand designed team might
have features that are rare in the search space. All of the games are played
against the MCTS based AI\@.

\section{Results}

\missingfigure{korelacni matice}
\missingfigure{grafy}

The results show that most participants consider our MCTS AI to be strong
enough to provide a challengem which is also proven by the 40\% winrate of
the participants. \todo{aktualizovat cisla}

Given the number of played games \todo{overit, ze jich mame dost} we can
rule out the AI having a better setup in most of the scenarios. Based on the
results, around 35\% \todo{aktualizovat cisla} of the games were balanced,
which means in the resulting 65\% one of the players had an advantage.

We consider this result to be positive as the games were generated without
any human intervention and weren't altered before running the experiment.
\todo{link na appendinx s instrukcema na pregenerovani experimentu}

Considering that 35\% of the generated games are balanced, this would allow
the algorithm to be used offline as is to aid design of game levels with
some manual checking of the resulting games.

If one were to design the encounter completely from scratch to be balanced,
it would be very difficult given the number of variables that need to be
optimized.

\section{TODO}

\begin{description}[align=right,labelwidth=3cm]
\item bud zminit ze by to slo delat online s nejaky additional checkem
\item nebo zlepsit fitness aby vyresila patologicke pripady z experimentu?
\end{description}


\chapter{Experiments}
\label{chapter05}

\section{Comparing AIs against each other}

To compare the strength of our AIs against each other, we simulated 1000 randomly generated games
and had the AIs play against each other. Both teams were generated at random and there was no step taken
to make them balanced. Instead, each AI played both sides of the match. See \autoref{tab:winrates} for results.

\begin{table}[h]
	\centering
	\begin{tabular}{|| c | c c c ||}
		\hline
		& MCTS & Rule & Random \\
		\hline\hline
		MCTS & N/A & 63\% & 88\% \\
		Rule & 37\%& N/A & 82\% \\
		Random & 12\% & 18\% & N/A \\
		\hline
	\end{tabular}
	\caption{Table showing win percentages over 1000 games between our different AI implementations.
	MCTS beats Random 88\% of the time, MCTS beats Rule 63\% of the time, and Rule beats Random 82\% of the time.}
	\label{tab:winrates}
\end{table}

As we can see, both MCTS and the Rule Based AI can beat our Random AI with a significant margin,
and MCTS can also consistently beat the Rule Based AI, despite playing for both generated teams.
From this we conclude that:

\begin{enumerate}
	\item A greedy aggressive strategy doesn't always win, which means the game mechanics are rich enough
	to reward strategic thinking.
	\item Thinking ahead (as MCTS does) provides enough value to be significant in terms of win rate.
\end{enumerate}

\section{Survey}
\label{survey}

We've conducted a small survey to verify the results of our encounter balancing algorithm.
The participants played 126 games in total and filled in a questionnaire after each game. Half of the encounters
in the games were completely generated by our algorithm, and in the other half we designed one team and let
the algorithm generate a suitable opponent.

\subsection{Participants}

The participants were all computer science students. All of them had at least
some experience with computer games, and were presented the mechanics of HexMage
before conducting the experiment.

\subsection{Experiment Design}

The goal of the survey is to measure two things. One, if the generated
encounters are actually balanced. And two, if the AI is strong enough to pose a
challenge to the player. We will measure the overall winrate of the players,
and also compare how the AI did in games the player considered balanced.

The experiment consists of 20 different games of HexMage. All of the games are
played on the same map that was hand-designed beforehand. This was to allow the
participants to better get familiar with the game and think ahead. The games
are structured so that each player has 2 mages, each with 2 abilities. This was
to reduce the cognitive overhead for the participants and allow them to more easily
adjust to the 20 completely different scenarios.

The first 10 of the 20 games had the player team hand designed, and the
opponent (played by the AI) generated with our PCG algorithm. The remaining 10
games had both teams generated with no manual tweaks or changes. Having some of
the games hand designed allows us to show that the PCG algorithm can balance
against constraints that aren't completely random. A hand designed team might
have features that are rare in the search space. All of the games are played
against the MCTS based AI\@ with a fixed number of iterations.

\subsection{Questions}

The questions were the following, answered on scale (1 - definitely no, to 7 - definitely yes), showing short codes for \autoref{tab:balance-corr} and \autoref{tab:difficulty-corr}

\begin{description}[]
	\item[Balanced:] The game balanced.
	\item[Challenge:] The game was challenging.
	\item[Unsure:] I wasn't sure who was going to win.
	\item[Smart:] The AI played smart.
	\item[Difficult:] The game was difficult.
	\item[Strategy:] The AI showed strategic thinking.
\end{description}

\subsection{Discussion of the Results}

As we can see in \autoref{tab:balance-corr}, the questions for balance strongly correlate.

\begin{table}[h]
	\centering
	\begin{tabular}{lrr}
		\toprule
		{} &  balanced &  unsure \\
		\midrule
		balanced &      1.00 &    \cellcolor{blue!25}0.73 \\
		unsure   &      0.73 &    1.00 \\
		\bottomrule
	\end{tabular}
	\caption{Correlation table between people who said the game was balanced and who were unsure about the result.}	
	\label{tab:balance-corr}
\end{table}

\begin{table}[h]
	\centering
	\begin{tabular}{lrrrr}
		\toprule
		{} &  challenge &  smart &  difficult &  strategy \\
		\midrule
		challenge &       1.00 &   0.61 &       0.46 &      0.63 \\
		smart     &       0.61 &   1.00 &       0.39 &      \cellcolor{blue!25}0.82 \\
		difficult &       0.46 &   0.39 &       1.00 &      0.39 \\
		strategy  &       0.63 &   0.82 &       0.39 &      1.00 \\
		\bottomrule
	\end{tabular}
	\caption{Correlation table between people who said the game was challenging, the AI was smart, the game was difficult to play and the AI showed strategic behavior.}
	\label{tab:difficulty-corr}
\end{table}


We can also take a look at the responses normalized to $0$ and $1$ where $1$ means
the response was at least $4$. Looking at \autoref{tab:norm-corr}, we can see that players
tend to consider the game difficult if they lost and vice versa.

\begin{table}[h]
	\centering
	\begin{tabular}{lrrrrrrr}
		\toprule
		{} &  balanced &  challenge &  unsure &  smart &  difficult &  strategy &   won \\
		\midrule
		balanced  &      1.00 &       0.35 &    0.53 &   0.21 &       0.07 &      0.22 &  0.15 \\
		challenge &      0.35 &       1.00 &    0.30 &   0.42 &       0.48 &      0.36 & -0.20 \\
		unsure    &      0.53 &       0.30 &    1.00 &   0.17 &       0.16 &      0.14 &  0.07 \\
		smart     &      0.21 &       0.42 &    0.17 &   1.00 &       0.49 &      0.79 & -0.32 \\
		difficult &      0.07 &       0.48 &    0.16 &   0.49 &       1.00 &      0.42 & \cellcolor{blue!25}-0.62 \\
		strategy  &      0.22 &       0.36 &    0.14 &   0.79 &       0.42 &      1.00 & -0.22 \\
		won       &      0.15 &      -0.20 &    0.07 &  -0.32 &      -0.62 &     -0.22 &  1.00 \\
		\bottomrule
	\end{tabular}
	\caption{Table of correlations after normalizing reponse values from 1--7 to 0--1 where 1 means the original response was at least 0.}
	\label{tab:norm-corr}
\end{table}

The total win rate of the players was only $38\%$, which shows that our MCTS~AI
is a formidable opponent. \autoref{tab:means} show the mean of the other responses.

\begin{table}[h!]
	\centering
	\begin{tabular}{lr}
		\toprule
		{} & mean \\ 
		\midrule
		balanced        &       0.60\% \\
		challenge       &       0.71\% \\
		unsure  &       0.57\% \\
		smart   &       0.78\% \\
		difficult       &       0.76\% \\
		strategy        &       0.78\% \\
		\bottomrule
	\end{tabular}
	\caption{Table of mean responses to all the questions in the questionnaire.}
	\label{tab:means}
\end{table}

Looking further at the results, in $4$ out of the $20$ games the players lost $100\%$ of the time,
and in $1$ out of the $20$ games the players won $100\%$ of the time. In the resulting $15$ games
there were both cases when a player won and when a player lost. We re-evaluated those games manually 
and found that in one case the game trully could not have been won, as one of the player mages always
got killed too early on and did not get to do any damage. On the other hand, the second game we tried
we won rather decisively, even though all of the participants in the experiment lost. The strategy to winning
strategy was however to play very defensively and calculate the opponents AP,
which is something the participants probably did not have a chance to do.

As for the third game, we also confirm it as impossible to win, as the opponent had a cheap ability and powerful ability
and could easily stay out of range while doing damage. The last of the always-lost games was also similar
in the sense that the opponent had a cheap ability he could use over and over again, and once he got a small
advantage he could push through to win the game without any resistance. While the games were close in terms
of HP at the end, there was no clear way to win.

In light of these results, there is a possibility for future work by incorporating an additional fitness
metric that makes sure the player can win the game, and not just that the games end up being close. However,
considering the players voted on $60\%$ of the games as balanced, we have definitely met our goal of
generating balanced encounters.

Note that all of the 20 games were generated without any human intervention, and the only input from the user
was to set the constants for attributes' range. The whole process of generating a balanced encounter is completely
automatic and replicable.


%%% Bibliography
\include{bibliography}

%%% Figures used in the thesis (consider if this is needed)
\listoffigures

%%% Tables used in the thesis (consider if this is needed)
%%% In mathematical theses, it could be better to move the list of tables to the beginning of the thesis.
\listoftables

%%% Abbreviations used in the thesis, if any, including their explanation
%%% In mathematical theses, it could be better to move the list of abbreviations to the beginning of the thesis.
\chapwithtoc{List of Abbreviations}

%%% Attachments to the bachelor thesis, if any. Each attachment must be
%%% referred to at least once from the text of the thesis. Attachments
%%% are numbered.
%%%
%%% The printed version should preferably contain attachments, which can be
%%% read (additional tables and charts, supplementary text, examples of
%%% program output, etc.). The electronic version is more suited for attachments
%%% which will likely be used in an electronic form rather than read (program
%%% source code, data files, interactive charts, etc.). Electronic attachments
%%% should be uploaded to SIS and optionally also included in the thesis on a~CD/DVD.
%%% Allowed file formats are specified in provision of the rector no. 23/2016.
\chapwithtoc{Attachments}

\openright
\end{document}
