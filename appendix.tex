\chapter{Appendix}
\label{appendix}

\section{Implementation overview}

The whole project is implemented in C\# on the .NET platform. As such, it is structured
into multiple Visual Studio \emph{projects} under a single \emph{solution}. A solution is a collection
of projects that are built together on the .NET platform. The projects are
the following:

\todo{popsat ze mame replaye}

\begin{description}
	\item[HexMage.Simulator] Contains all of the game's core logic, as well as implementation of the AIs
		and encounter balancing. This project is also compatible with Mono on Linux.
	\item[HexMage.Benchmarks] Small project that wraps all of the experiments in a command line interface.
		This project is also compatible with Mono on Linux.
	\item[HexMage.GUI] Contains the GUI of the game, which includes the arena, map editor, team generation
		and the questionnaire interface. It also has the ability to re-play games stored in replay files.
	\item[HexMage.Tests] Small project containing just a few unit tests for the most critical features.
\end{description}

There is also a directory structure under the \emph{data} directory that is used by the projects

\begin{description}
	\item[data/graphs] Contains GraphViz DOT \citep{graphviz} files for the MCTS tree. These are used mostly for debugging purposes and their generation can be turned on and off by defining a \emph{DOTGRAPH} compile time constant (see \emph{UctAlgorithm.cs} for details). Each iteration of MCTS is stored in a separate file.
	\item[data/images] Output directory for the generated image graphs from the \emph{graphs} folder.
	\item[data/manual-teams] Contains manually designed teams for the questionnaire. Each team is stored in its own JSON \citep{json} file.
	\item[data/questionnaire] Contains the generated encounters for our questionnaire. Each encounter is stored
	in a separate file. The file format is rather simple and described in \hyperref[dna-format]{the DNA file format section}.
	\item[data/replays] Contains replays recorded from games in JSON format. Replay recording can be turned on with the \emph{RecordReplays} command line argument. See \hyperref[cmd-args]{the command line section} for more details on how to provide command line arguments.
	\item[data/save-files] Contains saved DNA results in both the short DNA format and in JSON format (see \autoref{dna-format} for details). This folder is used by the search algorithm, which stores DNA with high enough fitness value.
\end{description}

There are also a few misc files in the data folder that serve mostly for data processing. Most notably the \path{generate.sh} file contains a Bash \citep{bash} script for generating MCTS tree images from the \path{data/graphs} folder.

\subsection{System Requirements}

All of the code is written for C\# 6.0. The \path{HexMage.GUI} project runs on Windows under .NET framework 4.5 (or newer). It also requires up to date graphics drivers, and optionally audio drivers when running with \texttt{--EnableAudio=true}. The experiments in \path{HexMage.Benchmarks} can be run on Linux under a recent version of Mono (tested on 4.x).

Compiling the whole solution requires Visual Studio 2015 or newer, as C\# 6.0 is required. The \path{HexMage.Benchmarks} and \path{HexMage.Simulator} projects can also be compiled under Mono on Linux. There is an attached \verb|Makefile| in the \path{HexMage.Benchmarks} project to simplify building under Mono.

\subsection{DNA file format}
\label{dna-format}

The DNA vectors are scored in simple text files. The format is as follows:

\begin{itemize}
	\item Each team is stored on a separate line.
	\item The line first contains two numbers, one for the number of mages in a team, and one for the number of abilities of each mage.
	\item Then follows the serialized vector for each mage in the order the mages were defined.
	\item Each mage is stored as his health, action points, and a list of abilities.
	\item Each ability is stored as damage, cost, range, cooldown, debuff, AOE.
	\item Debuff is stored as damage, action point damage and lifetime.
	\item AOE is stored as radius and the effect as if it was a debuff.
\end{itemize}

A reference implementation is in the \path{GenomeLoader.cs} file which contains both serialization and deserialization. Note that the starting positions of the mages are not stored in their DNA, as that information an attribute of the map on which the game is played. See \hyperref[map-format]{the Map file format section} for more details.

We also have an alternative file format for the DNA which can be seen in

\subsection{Map File Format}
\label{map-format}

The map is stored in a plain JSON file \citep{json}. An example can be found in the \path{data/map.json} file. The file defines the size of the map, whether each hex contains a wall or is empty, and the starting positions of each team. \todo{popsat tohle nekde driv?}. It is important to note that although there can be any number of starting positions on a given map, the team size must not exceed the number of starting positions. If the user tries to start a game with larger teams than the number of starting positions, the program will detect the error and not start the game.

\subsection{Command Line Arguments}
\label{cmd-args}

All of the important constants are configurable through command line arguments, both for the \path{HexMage.Benchmarks} and \path{HexMage.GUI} projects. The constant defaults are configured in \path{Constants.cs}. When the programs are run, the command line arguments are inspected and set respective values in the \path{Constants.cs} class via reflection. Adding a new command line argument is thus as simple as adding a new property to the \path{Constants.cs} file.

The program is then run as following:

\vspace{.5cm}
\texttt{HexMage.Benchmarks.exe --EnableSounds=true --TeamsPerGeneration=40}
\\

will set the \verb|EnableSounds| constant to \verb|true| and \verb|TeamsPerGeneration| to $40$. Constant types are automatically inferred by the reflection mechanism in .NET and can thus be specified without any additional boilerplate. The meaning of each constant is documented in the \path{Constants.cs} file. \todo{zdokumentovat konstanty ve zdrojaku}

\section{Running the Experiments}

There are multiple experiments that can be easily run through the command line via the \path{HexMage.Benchmarks.exe} file (or by running the \path{HexMage.Benchmarks} project in Visual Studio).

The following command line arguments are possible (in addition to the constants defined in \autoref{cmd-args}):

\begin{description}
	\item[mcts-measure-speed] Runs a MCTS speed benchmark which simply generates a random game and keeps playing it over and over again. The main purpose is for running MCTS itself under a profiler.
	
	\item[compare-ai] Runs an AI comparison benchmark. The types of AI that are selected are controlled via the \verb|MctsBenchType| constant (see \autoref{cmd-args}). The selected AIs are run against each other on random games and the resulting win rate is measured.
	
	\item[space-stats] Samples the search space at random points, looks around at neighbours and measures the ratio of upward/downward slopes with their respective fitness change.
	
	\todo{zadefinovat co je win rate}
\end{description}

When none of these arguments are specified, the encounter balancing algorithm is run on the questionnaire seed data (see \autoref{survey}).

\section{Implementation details}
\label{sec:impl-details}

We will now over go the internal structure of the project, the overall architecture and show a few examples on how to use the simulator library.

We chose C\# and the .NET platform since it provides a good balance of programmer productivity and performance. C\#, while being a high level language, still has the ability to control memory layout of objects to some extent. Most importantly, it has the \verb|struct| keyword which allows for objects which are allocated in-place and have copy semantics, contrast to Java, where all objects are dynamically allocated on the heap. This feature allows for compact representation of data and thus improves locality.

On the other hand, C\# also provides high level primitives in the form of LINQ and the Task Parallel Library \citep{tpl} (see \autoref{threading-tpl}). While developing, we also made heavy use of the advanced profiling tools provided for the .NET ecosystem, most notably dotTrace (\citet{dot-trace}). An example of where this is used is projectiles, where one can create a projectile and \verb|await| the task that represents the projectile hitting the target, and chain an animation afterwards.

\subsection{Performance}

We have spent a considerable amount of time optimizing the simulator to run as fast as possible. Most of the game state related data is stored in tightly packed arrays of structures. This was done to reduce the number of heap allocations and reduce overall pressure on the garbage collector \citep{highperf-dotnet}.

\todo{popsat vlastni sync context a prokladani vlaken pres framy}

\todo{vypichnout, ze simulator je rychlej}

Let us begin with \path{HexMage.GUI}.

\subsection{HexMage.GUI}

\todo{popsat proc vlastni gui knihovnu}

The GUI is written using the MonoGame \citep{monogame} library (version 3.5), which provides capabilities such as sprite rendering and user input handling. It also asset loading (textures, sprites, fonts) and compiles them together with the application. All of the assets can be found under \path{HexMage.GUI/Content}. Most of the graphics was done using the Aseprite \citep{aseprite} pixel art editor. Now onto the project structure.

We chose MonoGame because we wanted to use a higher level language (see \autoref{sec:impl-details}), and given the performance constraints, the only choice was between C\# with MonoGame and Java with libGDX \citep{libgdx}. We chose C\# as it has the tools for more compact data structures and better development tools. We didn't consider larger game engines such as Unity3D \citep{unity3d} since our game is rather simple, and we wanted to avoid spending most of the time trying to figure out how to do particular things in a big framework, but rather focus on the important parts.

The game is organized into \emph{Scenes}, where each Scene represents a single screen for the user. The scenes are the following (located under \path{HexMage.GUI/Scenes}):

\begin{description}
	\item[MapEditorScene.cs] Shows the map editor.
	\item[TeamSelectionScene.cs] Allows the user to generate teams and assign AIs to play the game.
	\item[ArenaScene.cs] Shows the current game, this is where the game is played.
	\item[QuestionnaireScene.cs] Serves as a UI for our survey. It shows the participants their progress.
\end{description}

Each scene contains an arbitrary amount of root \emph{Entities}. Each entity can have any amount of child Entities, and also any amount of \emph{Components} attached. Entities can also optionally have a \emph{Renderer} (subclass of \verb|IRenderer|) if they wish to be displayed to the user. Other than that, entities contain positional information, such as a relative position with respect to their parent, the order in which they are rendered, and a flags (i.e. \verb|Active| flag which determines if the entity is shown/updated on each frame). The Components are what give Entities most of their functionality, and can be an arbitrary class that extend from the \verb|Component| parent. The most important part of a Component is their \verb|Update| function, which gets called on every frame.

All of the user interface is implemented in terms of Entities and Components. We also provide some basic layout primitives, such as a \verb|VerticalLayout|, which lays out its children vertically as a list (with an optional spacing in between). We also implemented a small UI library in the form of clickable \emph{Buttons} which can have an arbitrary callback attached. The important part of the UI are the \path{GameBoardController} and \path{GameBoardRenderer} classes, which implement most of the logic for the game arena.

On top of the scene mechanism, we also provide basic user input handling in the form of the \verb|InputManager| class, and 2d camera in the \verb|Camera2D| class. This allows the user to zoom in/out and move around the map easily. The GUI also contains its own logging mechanism with different log levels.

\subsubsection{Threading and TPL}
\label{threading-tpl}

While the GUI is single threaded, we make use of the TPL for asynchronous task processing. This is implemented using a custom \verb|SynchronizationContext| subclass, which makes sure asynchronous callbacks are executed within the main event loop of the GUI. Implementation-wise it would have been simpler to use a worker thread and communicate with queues, but our approach provides much easier and flexible API for the developer. By implementing a global \verb|SynchronizationContext| we can make use of the .NET 4.x \verb|async| \citep{async} feature. 

\todo{zkontrolovat v kodu ze jsem neco neprehlid?}

\subsection{HexMage.Simulator}

The main core of our program is the simulator. It contains all of the game rules and mechanics and can be used as a standalone library. The main 