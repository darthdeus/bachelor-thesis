\chapter{Implementation Analysis}
\label{implementation-overview}

Since the whole project is implemented in \Csh{} on the .NET platform. As such, it is structured
into multiple Visual Studio \emph{projects} under a single \emph{solution}. A solution is a collection
of projects that are built together on the .NET platform. The projects are the following:

\begin{description}
	\item[HexMage.Simulator] Contains all of the game's core logic, as well as implementation of the AIs
	and encounter balancing. This project is also compatible with Mono on Linux.
	\item[HexMage.Benchmarks] Small project that wraps all of the experiments in a command line interface.
	This project is also compatible with Mono on Linux.
	\item[HexMage.GUI] Contains the GUI of the game, which includes the arena, map editor, team generation
	and the questionnaire interface. It also has the ability to re-play games stored in replay files.
	\item[HexMage.Tests] Small project containing just a few unit tests for the most critical features.
\end{description}

There is also a directory structure under the \emph{data} directory that is used by the projects

\begin{description}
	\item[data/graphs] Contains GraphViz DOT \citep{graphviz} files for the MCTS tree. These are used mostly for debugging purposes and their generation can be turned on and off by defining a \emph{DOTGRAPH} compile time constant (see \emph{UctAlgorithm.cs} for details). Each iteration of MCTS is stored in a separate file.
	\item[data/images] Output directory for the generated image graphs from the \emph{graphs} folder.
	\item[data/manual-teams] Contains manually designed teams for the questionnaire. Each team is stored in its own JSON \citep{json} file.
	\item[data/questionnaire] Contains the generated encounters for our questionnaire. Each encounter is stored
	in a separate file. The file format is rather simple and described in \hyperref[dna-format]{the DNA file format section}.
	\item[data/replays] Contains replays recorded from games in JSON format. Replay recording can be turned on with the \emph{RecordReplays} command line argument. See \hyperref[cmd-args]{the command line section} for more details on how to provide command line arguments.
	\item[data/save-files] Contains saved DNA results in both the short DNA format and in JSON format (see \autoref{dna-format} for details). This folder is used by the search algorithm, which stores DNA with high enough fitness value.
\end{description}

There are also a few misc files in the data folder that serve mostly for data processing. Most notably the \path{generate.sh} file contains a Bash \citep{bash} script for generating MCTS tree images from the \path{data/graphs} folder. We will now over go the internal structure of the project.

\section{HexMage.GUI}

The GUI is written using the MonoGame \citep{monogame} library (version 3.5), which provides capabilities such as sprite rendering and user input handling. It also asset loading (textures, sprites, fonts) and compiles them together with the application. All of the assets can be found under \path{HexMage.GUI/Content}. Most of the graphics was done using the Aseprite \citep{aseprite} pixel art editor. Now onto the project structure.

We chose MonoGame because we wanted to use a higher level language, and given the performance constraints, the only choice was between \Csh with MonoGame and Java with libGDX \citep{libgdx}. We chose \Csh as it has the tools for more compact data structures and better development tools. We didn't consider larger game engines such as Unity3D \citep{unity3d} since our game is rather simple, and we wanted to avoid spending most of the time trying to figure out how to do particular things in a big framework, but rather focus on the important parts.

The game is organized into \emph{Scenes}, where each Scene represents a single screen for the user. The scenes are the following (located under \path{HexMage.GUI/Scenes}):

\begin{description}
	\item[MapEditorScene.cs] Shows the map editor.
	\item[TeamSelectionScene.cs] Allows the user to generate teams and assign AIs to play the game.
	\item[ArenaScene.cs] Shows the current game, this is where the game is played.
	\item[QuestionnaireScene.cs] Serves as a UI for our survey. It shows the participants their progress.
\end{description}

Each scene contains an arbitrary amount of root \emph{Entities}. Each entity can have any amount of child Entities, and also any amount of \emph{Components} attached. Entities can also optionally have a \emph{Renderer} (subclass of \verb|IRenderer|) if they wish to be displayed to the user. Other than that, entities contain positional information, such as a relative position with respect to their parent, the order in which they are rendered, and a flags (i.e. \verb|Active| flag which determines if the entity is shown/updated on each frame). The Components are what give Entities most of their functionality, and can be an arbitrary class that extend from the \verb|Component| parent. The most important part of a Component is their \verb|Update| function, which gets called on every frame.

All of the user interface is implemented in terms of Entities and Components. We also provide some basic layout primitives, such as a \verb|VerticalLayout|, which lays out its children vertically as a list (with an optional spacing in between). We also implemented a small UI library in the form of clickable \emph{Buttons} which can have an arbitrary callback attached. The important part of the UI are the \path{GameBoardController} and \path{GameBoardRenderer} classes, which implement most of the logic for the game arena.

On top of the scene mechanism, we also provide basic user input handling in the form of the \verb|InputManager| class, and 2d camera in the \verb|Camera2D| class. This allows the user to zoom in/out and move around the map easily. The GUI also contains its own logging mechanism with different log levels.

\subsection{Threading and TPL}
\label{threading-tpl}

While the GUI is single threaded, we make use of the TPL for asynchronous task processing. This is implemented using a custom \verb|SynchronizationContext| subclass, which makes sure asynchronous callbacks are executed within the main event loop of the GUI. Implementation-wise it would have been simpler to use a worker thread and communicate with queues, but our approach provides much easier and flexible API for the developer. By implementing a global subclass of \verb|SynchronizationContext| we can make use of the .NET 4.x \verb|async| \citep{async} feature. 

Callbacks of \verb|async| methods are then automatically scheduled onto our custom event loop and are executed right before the \verb|Layout| and \verb|Update| lifecycle methods are called on the root entities (see \autoref{chap:prog-doc}). This also allows actions to be interleaved with the regular game loop without having to worry about locks and shared mutable state. We still have the ability to execute tasks on a thread pool using the \verb|Task.Run| method from the TPL \citep{tpl}, but at the same time we can specify the callback to run on the game event loop using the await \verb|await| thanks to our custom \verb|SynchronizationContext|.

\section{HexMage.Simulator}
\label{sec:simulator}

The main core of our program is the simulator. It contains all of the game rules and mechanics and can be used as a standalone library. The main class is \verb|GameInstance| which encapsulates the game state. It delegates most of its functionality to the following classes:

\begin{description}
	\item[Map] Contains all of the logic related to the map itself, such as whether each hex is empty or a wall, and also starting point information
	
	\item[Pathfinder] Calculates and caches pathfinding and visibility information for the given map.
	
	\item[MobManager] Stores immutable information about the mages and their abilities. For example maximum amounts of HP and action points, ability definitions, etc.
	
	\item[GameState] Contains state information which changes as the game is played. This contains the current values, such as current HP, action points and position on the map for each mage.
	
	\item[TurnManager] Encapsulates all of the logic regarding turns and rounds.
	
	\item[GameEventHub] Handles the main game loop and handles switching turns between players, and detects whether the game has finished.
\end{description}

\section{Performance}

We have spent a considerable amount of time optimizing the simulator to run as fast as possible. Most of the game state related data is stored in tightly packed arrays of structures. This was done to reduce the number of heap allocations and reduce overall pressure on the garbage collector \citep{highperf-dotnet}. It also improves copy time of the game state, which happens during each iteration of the MCTS algorithm.

With \verb|DefensiveMove| generation enabled, MCTS can run around $30000$ iterations per second, and the simulator running around $150000$ actions per second (on the Intel i7-5820k Processor \citep{intel-ark}). Note that most of the time is spent on generating high level actions. When \verb|DefensiveMove| actions are disabled, MCTS can run nearly twice as fast since it can avoid re-calculating which positions on the map are safe. Improving the high level action generation is a possibility for future work. More elaborate heuristics could allow MCTS to handle larger map sizes and perform even better as the search depth is limited by the number of iterations it can run within a given amount of time.


\subsection{Map implementation}

Since the map is composed of hexagons, some special care was taken to implement things properly. Most of the algorithms such as calculating visibility and drawing layout were inspired by Amit Patel's Hexagonal Grids article \citep{hexagons}.

We use a 2-dimensional array to store the hexagonal map (class \verb|HexMap<T>|) and restrict the valid coordinates to only those with a distance of $N$ from the center, which makes the resulting map have a hexagonal shape as well (with a radius of $N$).

\subsection{Game Events and AI Controllers}

Our main game loop is encapsulated by the \verb|GameEventHub| class. It does not however pick which action each player is doing, that is the job of a \verb|IMobController|. Each AI we implemented has its own implementation of the \verb|IMobController| interface, and there is also one for the human player called \verb|PlayerController|. This allows us to write the game loop without knowing who is going to play the game. The responsibility of the controller is only to dispatch proper actions during its turn.

Also worth noting that since we're not using threading in the UI, but rather asynchronously queued tasks in the event loop, we can't simply block and wait for the \verb|PlayerController| to respond with a given action, as that would freeze the game UI. Instead, we provide an asynchronous game loop in the form of \verb|GameEventHub.SlowMainLoop|. It handles turn management automatically and all the programmer has to provide is an implementation of the \verb|IMobController| (see \autoref{chap:prog-doc}).