\chapwithtoc{Introduction}

An increasing number of computer games is using procedural content generation
(PCG) as one of their core mechanics. This is in different contexts, most
commonly for generating new levels (e.g. Diablo \citep{diablo}). Occasionally games
even generate player collectible items (e.g. Borderlands \citep{borderlands}). However,
there has not been much research on the use of procedural generation for
balancing encounters in RPG games. By this we mean procedurally generating
enemies that can be defeated by the player, but pose a challenge. A crucial
criteria here is that the balance is not simply achieved by creating the enemy
as an exact clone of the player, but rather explore the search-space to find an
enemy that is not only balanced, but also different from the player.

One particular application for this kind of PCG is automatic difficulty
adjustments based on the player's skill. Another possible use could be
automatic generation of new and unique enemies based on given constraints,
which is the approach we chose in this thesis.

We have created a custom game with mechanics that are simple enough to simulate
quickly, yet flexible enough to represent a large search space. There are two
teams that fight in a hexagonal arena, each consists of a small number of
player controlled characters (mages), and each mage has a small number of
abilities. In each turn the player has control over one of his characters, and
both move around the map and cast spells, in any order he wishes. The only
limit is the number of action points the character has available, which are
consumed both by movement and ability usage. The side that first eliminates the
opposing team wins.

All information is visible to all players, and all actions are completely
deterministic. There is no time limit for the player action, which means the
player could theoretically calculate a perfect move given enough time.

The goal of this thesis is to explore PCG options for balancing encounters in
turn-based RPG-like games. We design a simple game with flexible mechanics, and
build an AI that can be used to approximate the player. We then use evolution strategies
to generate opponents of just the right difficulty for the given player team, using AI vs
AI combat as a fitness function.

\subsection*{Organization}

In Chapter 1 we begin by defining the scope of the work, our general approach,
and list related work. Chapter 2 follows by exploring our game mechanics in
detail and explaining the choices behind them. Next in Chapter 3 we will go
over our different choices for implementing the AI.

Chapter 4 describes our approach to generating the encounters, continuing with an overview of the implementation in Chapter 5.  Next, the expriments are described in Chapter 6 with a conclusion in Chapter 7. Lastly, the Appendix contains user and programmer documentation, as well as description of the used file formats.
